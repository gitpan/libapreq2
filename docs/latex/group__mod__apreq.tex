\section{Apache-2 Filter Module}
\label{group__mod__apreq}\index{Apache-2 Filter Module@{Apache-2 Filter Module}}
mod\_\-apreq.c: Apache-2 filter module. 
\subsection*{Defines}
\begin{CompactItemize}
\item 
\index{APREQ_MODULE_NAME@{APREQ\_\-MODULE\_\-NAME}!mod_apreq@{mod\_\-apreq}}\index{mod_apreq@{mod\_\-apreq}!APREQ_MODULE_NAME@{APREQ\_\-MODULE\_\-NAME}}
\#define {\bf APREQ\_\-MODULE\_\-NAME}\ \char`\"{}APACHE2\char`\"{}\label{group__mod__apreq_a19}

\item 
\index{APREQ_MODULE_MAGIC_NUMBER@{APREQ\_\-MODULE\_\-MAGIC\_\-NUMBER}!mod_apreq@{mod\_\-apreq}}\index{mod_apreq@{mod\_\-apreq}!APREQ_MODULE_MAGIC_NUMBER@{APREQ\_\-MODULE\_\-MAGIC\_\-NUMBER}}
\#define {\bf APREQ\_\-MODULE\_\-MAGIC\_\-NUMBER}\ 20040324\label{group__mod__apreq_a20}

\end{CompactItemize}


\subsection{Detailed Description}
mod\_\-apreq.c: Apache-2 filter module.

mod\_\-apreq.c provides an input filter for using libapreq2 (and allow its parsed data structures to be shared) within the Apache-2 webserver. Using it, libapreq2 works properly in every phase of the HTTP request, from translation handlers  to output filters, and even for subrequests / internal redirects.

After installing mod\_\-apreq, be sure your webserver's httpd.conf activates it on startup with a Load\-Module directive: \small\begin{alltt}{\tt 

     LoadModule modules/mod_apreq.so

 }\end{alltt}\normalsize 
 Normally the installation process triggered by '\% make install' will make the necessary changes to httpd.conf for you.

XXX describe normal operation, effects of apreq\_\-config\_\-t settings, etc. 