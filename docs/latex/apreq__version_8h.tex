\section{apreq\_\-version.h File Reference}
\label{apreq__version_8h}\index{apreq_version.h@{apreq\_\-version.h}}
Versioning API for libapreq. 


{\tt \#include \char`\"{}apreq.h\char`\"{}}\par
{\tt \#include \char`\"{}apr\_\-version.h\char`\"{}}\par
\subsection*{Defines}
\begin{CompactItemize}
\item 
\#define {\bf APREQ\_\-MAJOR\_\-VERSION}\ 2
\item 
\#define {\bf APREQ\_\-MINOR\_\-VERSION}\ 0
\item 
\#define {\bf APREQ\_\-PATCH\_\-VERSION}\ 7
\item 
\#define {\bf APREQ\_\-IS\_\-DEV\_\-VERSION}
\item 
\#define {\bf APREQ\_\-VERSION\_\-STRING}
\item 
\#define {\bf APREQ\_\-IS\_\-DEV\_\-STRING}\ \char`\"{}-dev\char`\"{}
\end{CompactItemize}
\subsection*{Functions}
\begin{CompactItemize}
\item 
{\bf void} {\bf apreq\_\-version} ({\bf apr\_\-version\_\-t} $\ast$pvsn)
\item 
const char $\ast$ {\bf apreq\_\-version\_\-string} ({\bf void})
\end{CompactItemize}


\subsection{Detailed Description}
Versioning API for libapreq.





 There are several different mechanisms for accessing the version. There is a string form, and a set of numbers; in addition, there are constants which can be compiled into your application, and you can query the library being used for its actual version.

Note that it is possible for an application to detect that it has been compiled against a different version of libapreq by use of the compile-time constants and the use of the run-time query function.

libapreq version numbering follows the guidelines specified in:

{\tt http://apr.apache.org/versioning.html}



\subsection{Define Documentation}
\index{apreq_version.h@{apreq\_\-version.h}!APREQ_IS_DEV_STRING@{APREQ\_\-IS\_\-DEV\_\-STRING}}
\index{APREQ_IS_DEV_STRING@{APREQ\_\-IS\_\-DEV\_\-STRING}!apreq_version.h@{apreq\_\-version.h}}
\subsubsection{\setlength{\rightskip}{0pt plus 5cm}\#define APREQ\_\-IS\_\-DEV\_\-STRING\ \char`\"{}-dev\char`\"{}}\label{apreq__version_8h_a5}


Internal: string form of the \char`\"{}is dev\char`\"{} flag \index{apreq_version.h@{apreq\_\-version.h}!APREQ_IS_DEV_VERSION@{APREQ\_\-IS\_\-DEV\_\-VERSION}}
\index{APREQ_IS_DEV_VERSION@{APREQ\_\-IS\_\-DEV\_\-VERSION}!apreq_version.h@{apreq\_\-version.h}}
\subsubsection{\setlength{\rightskip}{0pt plus 5cm}\#define APREQ\_\-IS\_\-DEV\_\-VERSION}\label{apreq__version_8h_a3}


This symbol is defined for internal, \char`\"{}development\char`\"{} copies of libapreq. This symbol will be undef'd for releases. \index{apreq_version.h@{apreq\_\-version.h}!APREQ_MAJOR_VERSION@{APREQ\_\-MAJOR\_\-VERSION}}
\index{APREQ_MAJOR_VERSION@{APREQ\_\-MAJOR\_\-VERSION}!apreq_version.h@{apreq\_\-version.h}}
\subsubsection{\setlength{\rightskip}{0pt plus 5cm}\#define APREQ\_\-MAJOR\_\-VERSION\ 2}\label{apreq__version_8h_a0}


major version  Major API changes that could cause compatibility problems for older programs such as structure size changes. No binary compatibility is possible across a change in the major version. \index{apreq_version.h@{apreq\_\-version.h}!APREQ_MINOR_VERSION@{APREQ\_\-MINOR\_\-VERSION}}
\index{APREQ_MINOR_VERSION@{APREQ\_\-MINOR\_\-VERSION}!apreq_version.h@{apreq\_\-version.h}}
\subsubsection{\setlength{\rightskip}{0pt plus 5cm}\#define APREQ\_\-MINOR\_\-VERSION\ 0}\label{apreq__version_8h_a1}


Minor API changes that do not cause binary compatibility problems. Should be reset to 0 when upgrading APREQ\_\-MAJOR\_\-VERSION \index{apreq_version.h@{apreq\_\-version.h}!APREQ_PATCH_VERSION@{APREQ\_\-PATCH\_\-VERSION}}
\index{APREQ_PATCH_VERSION@{APREQ\_\-PATCH\_\-VERSION}!apreq_version.h@{apreq\_\-version.h}}
\subsubsection{\setlength{\rightskip}{0pt plus 5cm}\#define APREQ\_\-PATCH\_\-VERSION\ 7}\label{apreq__version_8h_a2}


patch level \index{apreq_version.h@{apreq\_\-version.h}!APREQ_VERSION_STRING@{APREQ\_\-VERSION\_\-STRING}}
\index{APREQ_VERSION_STRING@{APREQ\_\-VERSION\_\-STRING}!apreq_version.h@{apreq\_\-version.h}}
\subsubsection{\setlength{\rightskip}{0pt plus 5cm}\#define APREQ\_\-VERSION\_\-STRING}\label{apreq__version_8h_a4}


{\bf Value:}

\footnotesize\begin{verbatim}APR_STRINGIFY(APREQ_MAJOR_VERSION) "." \
     APR_STRINGIFY(APREQ_MINOR_VERSION) "." \
     APR_STRINGIFY(APREQ_PATCH_VERSION) \
     APREQ_IS_DEV_STRING\end{verbatim}\normalsize 
The formatted string of libapreq's version 

\subsection{Function Documentation}
\index{apreq_version.h@{apreq\_\-version.h}!apreq_version@{apreq\_\-version}}
\index{apreq_version@{apreq\_\-version}!apreq_version.h@{apreq\_\-version.h}}
\subsubsection{\setlength{\rightskip}{0pt plus 5cm}{\bf void} apreq\_\-version ({\bf apr\_\-version\_\-t} $\ast$ {\em pvsn})}\label{apreq__version_8h_a6}


Return libapreq's version information information in a numeric form.\begin{Desc}
\item[Parameters: ]\par
\begin{description}
\item[{\em 
pvsn}]Pointer to a version structure for returning the version information. \end{description}
\end{Desc}
\index{apreq_version.h@{apreq\_\-version.h}!apreq_version_string@{apreq\_\-version\_\-string}}
\index{apreq_version_string@{apreq\_\-version\_\-string}!apreq_version.h@{apreq\_\-version.h}}
\subsubsection{\setlength{\rightskip}{0pt plus 5cm}const char$\ast$ apreq\_\-version\_\-string ({\bf void})}\label{apreq__version_8h_a7}


Return libapreq's version information as a string. 