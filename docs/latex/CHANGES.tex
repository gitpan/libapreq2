\subsection{Changes with libapreq2-2.01-dev}\label{v2_01_dev}
\begin{itemize}
\item November 7, 2003 - build system [joes]\end{itemize}


Skip Apache::Test tests in env/ when Apache::Test is unavailable. This allows the C API to be build and installed without requiring Apache::Test (it is still a requirement for compiling the perl glue).

\begin{itemize}
\item November 7, 2003 - C API {\bf mod\_\-apreq.c} [joes]\end{itemize}


Parser errors were creeping into the return value of apreq\_\-filter, which breaks the \char`\"{}transparent tee\char`\"{} paradigm. This caused bogus \char`\"{}400 Bad Request\char`\"{} responses (first reported by Vladimir Dudo) to occur when libapreq2 was used by an output filter during a GET request (handled by apache2's default handler). The test suite has been updated accordingly.

\begin{itemize}
\item October 26, 2003 - C API [joes]\end{itemize}


Incorporate libapreq\_\-cgi into libapreq2 as the default environment, and add apreq\_\-env\_\-t and initializer {\bf apreq\_\-env\_\-module()} to manage the environment at runtime (determining the environment at load-time was problematic on non-ELF systems).\subsection{Changes with libapreq2-2.00-dev}\label{v2_0_0}
\begin{itemize}
\item October 24, 2003 - C API: libapreq\_\-cgi.c [randyk, joes]\end{itemize}


CGI environment defined by env/libapreq\_\-cgi.c is functional (with tests added to env/t). This library may soon be incorporated directly into libapreq2 as a default enviroment.

\begin{itemize}
\item October 23, 2003 - C API: {\bf mod\_\-apreq.c} [joes]\end{itemize}


Added ctx-$>$saw\_\-eos to ensure we don't read from upstream filters after receiving an eos bucket. Otherwise it was possible for two eos buckets to appear when a prefetch read is involved, which breaks other modules like mod\_\-proxy. This bug was uncovered by Philippe Chiasson. mod\_\-apreq's apreq\_\-env\_\-majic\_\-number bumped to reflect the added fixes.

\begin{itemize}
\item October 17, 2003 - configure: --enable-perl-glue [joes]\end{itemize}


The --enable-perl-glue option integrates the perl glue into the normal Unix build cycle. It is disabled by default, but is silently reenabled if the user configures the source tree via Makefile.PL.

\begin{itemize}
\item October 14, 2003 - C API [joes]\end{itemize}


Added {\bf apreq\_\-header\_\-attribute()} and fixed mfd parser to allow \char`\"{}charset\char`\"{} attribute to appear in the Content-Type header. Sven Geisler points out that Opera 7.20 does generate such headers.

\begin{itemize}
\item October 14, 2003 - C API [joes]\end{itemize}


Added versioning API following {\tt http://apr.apache.org/versioning.html} apreq\_\-env renamed apreq\_\-env\_\-name, and apreq\_\-env\_\-magic\_\-number added to provide versioning for environments (modules). The header files are now installed to \char`\"{}include/apreq2\char`\"{}, and the library is renamed \char`\"{}libapreq2\char`\"{}. Also added an apreq2-config script based on apu-config.

\begin{itemize}
\item October 8, 2003 - configure: static {\bf mod\_\-apreq.c} [Bojan Smojver, joes]\end{itemize}


Add --with-apache2-src configure option, along with --with-apr-config and --with-apu-config, and provide support for compiling mod\_\-apreq into httpd as a static apache module.

\begin{itemize}
\item October 1, 2003 - C API: {\bf mod\_\-apreq.c} [joes]\end{itemize}


Support for internal redirects added to the mod\_\-apreq filter. This ensures any POST data prefetched in the main request gets passed along to the subrequest handler(s).

\begin{itemize}
\item July 18, 2003 - C bugfix: apreq\_\-decode [Graham Clark]\end{itemize}


If the source and destination strings are represented by the same pointer - e.g. if called as apreq\_\-unescape(s) - string s is modified incorrectly in general. Patch includes new unit test.

\begin{itemize}
\item July 16, 2003 - Perl API [joes]\end{itemize}


Added \$req-$>$parse, \$req-$>$status, \& \char`\"{}preparse\char`\"{} logic to \$req-$>$param \& \$req-$>$upload.

\begin{itemize}
\item July 16, 2003 - C API [joes]\end{itemize}


Added \char`\"{}preparse\char`\"{} logic to apreq\_\-params \& apreq\_\-uploads to bring behavior in line with libapreq-1.x.

\begin{itemize}
\item July 15, 2003 - C API [joes]\end{itemize}


Dropped param-$>$charset. Make apreq\_\-brigade\_\-concat public, so mod\_\-apreq can use it for its ctx-$>$spool brigade.

\begin{itemize}
\item July 14, 2003 - Documentation [joes]\end{itemize}


Updated Cookie\_\-pod to reflect API changes over v1.X.

\begin{itemize}
\item June 30, 2003 - Documentation [joes]\end{itemize}


Added doxygen links to Apache::Request and Apache::Cookie perl docs.

\begin{itemize}
\item June 30, 2003 - C API [joes]\end{itemize}


Added apreq\_\-copy\_\-brigade(bb) to {\bf apreq.h}.

\begin{itemize}
\item June 27, 2003 - C API [joes]\end{itemize}


The new filter-based design required a complete departure from libapreq-1.X codebase. libapreq-2 is based solely on APR, and to be fully functional, requires a supporting environment similar to Apache-2. A person wishing to port libapreq-2 to a new environment needs to provide definitions for the declarations in {\bf apreq\_\-env.h}.

\begin{itemize}
\item June 27, 2003 - Perl API [joes]\end{itemize}


Aggregates are always collected into an APR::Table-based package. New table packages: Apache::Cookie::Table, Apache::Request::Table, and Apache::Upload::Table.

\begin{itemize}
\item June 27, 2003 - Perl API [joes]\end{itemize}


Apache::Cookie-$>$fetch now requires an \char`\"{}environment\char`\"{} argument (\$r). Its return value is blessed into the Apache::Cookie::Jar class.

\begin{itemize}
\item June 27, 2003 - Perl API [joes]\end{itemize}


Two new request lookup functions:\begin{enumerate}
\item \$req-$>$args - param lookup using only the query string\item \$req-$>$body - param lookup using only the POST data\end{enumerate}
