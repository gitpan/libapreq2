

\section{Changes with libapreq2-2.03-dev (released June 12, 2004)}\label{v2_03_dev}


\begin{CompactItemize}
\item 
C API [joes] \char`\"{}Objectify\char`\"{} cookie/jar API: s/apreq\_\-(make$|$serialize)\_\-cookie/apreq\_\-cookie\_\-\$1/ and reordering args so the cookie/jar object is always the first argument. Macros added to provide source-compatibility with the old names.\end{CompactItemize}
\begin{CompactItemize}
\item 
Perl API [joes] Added \$upload-$>$slurp(\$data), which reads the contents of the file upload \char`\"{}\$upload\char`\"{} into the scalar \char`\"{}\$data\char`\"{}.\end{CompactItemize}
\begin{CompactItemize}
\item 
C API [joes, randyk] apreq\_\-run\_\-(hook$|$parser) are macros, so they are capitalized now. Fixed {\bf apreq\_\-params\_\-as\_\-string}() {\rm (p.\,\pageref{group__params_a9})} and added {\bf apreq\_\-params\_\-as\_\-array}() {\rm (p.\,\pageref{group__params_a8})}. Reworked definitions of APREQ\_\-DECLARE\_\-HOOK, APREQ\_\-DECLARE\_\-PARSER  and apreq\_\-(parser$|$hook)\_\-t, hopefully to be more Win32 friendly. Also updated the documentation.\end{CompactItemize}
\begin{CompactItemize}
\item 
C API [joes] Compensate for a missing CRLF in empty file upload block, which  actually complies with RFC 2046 Section 5.1.1. Konqueror (version unknown) and Mozilla 0.9.7 are known to emit such blocks.\end{CompactItemize}
\begin{CompactItemize}
\item 
Perl API [joes] \$req-$>$upload() in list context failed to filter out non-uploads. Also \$req-$>$upload(\char`\"{}nonexistent-key-name\char`\"{}) segfaults.\end{CompactItemize}
\begin{CompactItemize}
\item 
Perl test suite t/TEST.PL must run parent class' pre\_\-configure to get the configuration right\end{CompactItemize}
\begin{CompactItemize}
\item 
C API [joes] {\bf apreq\_\-brigade\_\-concat}() {\rm (p.\,\pageref{group__params_a16})} wasn't supplying the final EOS bucket to large brigades ($>$256K), which somtimes caused the prefetch  loop in mod\_\-apreq.c's {\bf apreq\_\-filter}() to hang.\end{CompactItemize}
\begin{CompactItemize}
\item 
Documentation [joes] CHANGES file reformatted, removing dates \& other clutter  as Stas suggests.\end{CompactItemize}
\begin{CompactItemize}
\item 
C API [joes] Rewrote {\bf cgi\_\-read}() in apreq\_\-env.c and reworked mod\_\-apreq.c  to enforce {\bf apreq\_\-env\_\-max\_\-body}() {\rm (p.\,\pageref{group__ENV_a10})} settings.\end{CompactItemize}
\begin{CompactItemize}
\item 
C API [joes] Fixed bug in url\_\-parser code- missing context brigade was needed to track key-value pairs which span multiple buckets.\end{CompactItemize}
\begin{CompactItemize}
\item 
C API [joes] API modifications: removed struct apreq\_\-cfg\_\-t, adding new apreq\_\-env hooks max\_\-body, max\_\-brigade\_\-len, and temp\_\-dir. Folded apreq\_\-parsers.h into {\bf apreq\_\-params.h} and modified the arguments  to apreq\_\-run\_\-parser() and apreq\_\-run\_\-hook(). Renamed  {\bf apreq\_\-parser\_\-t} {\rm (p.\,\pageref{structapreq__parser__t})}'s content\_\-type as enctype and apreq\_\-copy\_\-brigade() as {\bf apreq\_\-brigade\_\-copy}() {\rm (p.\,\pageref{group__Utils_a22})}. These changes make libapreq2.so.2.0.5 incompatible with earlier  versions.\end{CompactItemize}
\begin{CompactItemize}
\item 
Perl API [stas] Include ppport.h from blead-perl to support older perls. Add a proper support for ithreads.\end{CompactItemize}
\begin{CompactItemize}
\item 
C API [Swen Schillig, joes] Fixed bug in calculation of Netscape cookie expiration dates. apr\_\-time\_\-t is measured in microseconds, not seconds, which threw off the arithmetic; apr\_\-time\_\-from\_\-sec was needed for the conversion.\end{CompactItemize}
\begin{CompactItemize}
\item 
C API [Max Kellermann] Fix segfault caused by invalid \%-escape sequence in query string.\end{CompactItemize}
\section{Changes with libapreq2-2.02-dev (released Nov 15, 2003)}\label{v2_02_dev}


\begin{CompactItemize}
\item 
Perl API [joes] Fix bogus pool/cookie initializers in Apache::Cookie::set\_\-attr(), which caused Apache::Cookie::new to segfault. Bug first reported to modperl list by Wolfgang Kubens.\end{CompactItemize}
\section{Changes with libapreq2-2.01-dev (released Nov 10, 2003)}\label{v2_01_dev}


\begin{CompactItemize}
\item 
build system [joes] Skip Apache::Test tests in env/ when Apache::Test is unavailable. This allows the C API to be build and installed without requiring Apache::Test (it is still a requirement for compiling the perl glue).\end{CompactItemize}
\begin{CompactItemize}
\item 
C API mod\_\-apreq.c [joes] Parser errors were creeping into the return value of apreq\_\-filter, which breaks the \char`\"{}transparent tee\char`\"{} paradigm. This caused bogus \char`\"{}400 Bad Request\char`\"{} responses (first reported by Vladimir Dudo)  to occur when libapreq2 was used by an output filter during a GET  request (handled by apache2's default handler). The test suite  has been updated accordingly.\end{CompactItemize}
\begin{CompactItemize}
\item 
C API [joes] Incorporate libapreq\_\-cgi into libapreq2 as the default environment, and add {\bf apreq\_\-env\_\-t} {\rm (p.\,\pageref{structapreq__env__t})} and initializer {\bf apreq\_\-env\_\-module}() {\rm (p.\,\pageref{group__ENV_a12})} to manage the environment at runtime (determining the environment at load-time was problematic on non-ELF systems).\end{CompactItemize}
\section{Changes with libapreq2-2.00-dev (Oct 25, 2003)}\label{v2_0_0}


\begin{CompactItemize}
\item 
C API: libapreq\_\-cgi.c [randyk, joes] CGI environment defined by env/libapreq\_\-cgi.c is functional (with tests added to env/t). This library may soon be incorporated directly into libapreq2 as a default enviroment.\end{CompactItemize}
\begin{CompactItemize}
\item 
C API: mod\_\-apreq.c [joes] Added ctx-$>$saw\_\-eos to ensure we don't read from upstream filters after receiving an eos bucket. Otherwise it was possible for two eos buckets to appear when a prefetch read is involved, which breaks other modules like mod\_\-proxy. This bug was uncovered by Philippe Chiasson. mod\_\-apreq's apreq\_\-env\_\-majic\_\-number bumped to reflect the added fixes.\end{CompactItemize}
\begin{CompactItemize}
\item 
configure: --enable-perl-glue [joes] The --enable-perl-glue option integrates the perl glue into the  normal Unix build cycle. It is disabled by default, but is silently  reenabled if the user configures the source tree via Makefile.PL.\end{CompactItemize}
\begin{CompactItemize}
\item 
C API [joes] Added {\bf apreq\_\-header\_\-attribute}() {\rm (p.\,\pageref{group__Utils_a23})} and fixed mfd parser to allow  \char`\"{}charset\char`\"{} attribute to appear in the Content-Type header. Sven Geisler points out that Opera 7.20 does generate such headers.\end{CompactItemize}
\begin{CompactItemize}
\item 
C API [joes] Added versioning API following {\tt http://apr.apache.org/versioning.html} apreq\_\-env renamed apreq\_\-env\_\-name, and apreq\_\-env\_\-magic\_\-number added to provide versioning for environments (modules). The header files  are now installed to \char`\"{}include/apreq2\char`\"{}, and the library is renamed  \char`\"{}libapreq2\char`\"{}. Also added an apreq2-config script based on apu-config.\end{CompactItemize}
\begin{CompactItemize}
\item 
configure: static mod\_\-apreq.c [Bojan Smojver, joes] Add --with-apache2-src configure option, along with --with-apr-config and --with-apu-config, and provide support for compiling mod\_\-apreq  into httpd as a static apache module.\end{CompactItemize}
\begin{CompactItemize}
\item 
C API: mod\_\-apreq.c [joes] Support for internal redirects added to the mod\_\-apreq filter. This ensures any POST data prefetched in the main request  gets passed along to the subrequest handler(s).\end{CompactItemize}
\begin{CompactItemize}
\item 
C bugfix: apreq\_\-decode [Graham Clark] If the source and destination strings are represented by the same pointer - e.g. if called as apreq\_\-unescape(s) - string s is modified incorrectly in general. Patch includes new unit test.\end{CompactItemize}
\begin{CompactItemize}
\item 
Perl API [joes] Added \$req-$>$parse, \$req-$>$status, \& \char`\"{}preparse\char`\"{} logic  to \$req-$>$param \& \$req-$>$upload.\end{CompactItemize}
\begin{CompactItemize}
\item 
C API [joes] Added \char`\"{}preparse\char`\"{} logic to apreq\_\-params \& apreq\_\-uploads to bring behavior in line with libapreq-1.x.\end{CompactItemize}
\begin{CompactItemize}
\item 
C API [joes] Dropped param-$>$charset. Make apreq\_\-brigade\_\-concat public, so mod\_\-apreq can use it for its ctx-$>$spool brigade.\end{CompactItemize}
\begin{CompactItemize}
\item 
Documentation [joes] Updated Cookie\_\-pod to reflect API changes over v1.X.\end{CompactItemize}
\begin{CompactItemize}
\item 
Documentation [joes] Added doxygen links to Apache::Request and Apache::Cookie  perl docs.\end{CompactItemize}
\begin{CompactItemize}
\item 
C API [joes] Added apreq\_\-copy\_\-brigade(bb) to {\bf apreq.h}.\end{CompactItemize}
\begin{CompactItemize}
\item 
C API [joes] The new filter-based design required a complete  departure from libapreq-1.X codebase. libapreq-2 is based solely on APR, and to be fully functional,  requires a supporting environment similar to Apache-2. A person wishing to port libapreq-2 to a new environment needs to provide definitions for the declarations in {\bf apreq\_\-env.h}.\end{CompactItemize}
\begin{CompactItemize}
\item 
Perl API [joes] Aggregates are always collected into an APR::Table-based package. New table packages: Apache::Cookie::Table, Apache::Request::Table, and Apache::Upload::Table.\end{CompactItemize}
\begin{CompactItemize}
\item 
Perl API [joes] Apache::Cookie-$>$fetch now requires an \char`\"{}environment\char`\"{} argument (\$r). Its return value is blessed into the Apache::Cookie::Jar class.\end{CompactItemize}
\begin{CompactItemize}
\item 
Perl API [joes] Two new request lookup functions:\begin{enumerate}
\item 
\$req-$>$args - param lookup using only the query string\item 
\$req-$>$body - param lookup using only the POST data\end{enumerate}
\end{CompactItemize}
