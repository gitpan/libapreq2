\section{Cookies (request and response)}
\label{group__cookies}\index{Cookies (request and response)@{Cookies (request and response)}}
\subsection*{Data Structures}
\begin{CompactItemize}
\item 
struct {\bf apreq\_\-cookie\_\-t}
\item 
struct {\bf apreq\_\-jar\_\-t}
\end{CompactItemize}
\subsection*{Defines}
\begin{CompactItemize}
\item 
\#define {\bf APREQ\_\-COOKIE\_\-VERSION\_\-DEFAULT}\ APREQ\_\-COOKIE\_\-VERSION\_\-NETSCAPE
\item 
\#define {\bf APREQ\_\-COOKIE\_\-MAX\_\-LENGTH}\ 4096
\item 
\#define {\bf apreq\_\-value\_\-to\_\-cookie}(ptr)
\item 
\index{apreq_cookie_name@{apreq\_\-cookie\_\-name}!cookies@{cookies}}\index{cookies@{cookies}!apreq_cookie_name@{apreq\_\-cookie\_\-name}}
\#define {\bf apreq\_\-cookie\_\-name}(c)\ ((c) $\rightarrow$ v.name)\label{group__cookies_a16}

\item 
\index{apreq_cookie_value@{apreq\_\-cookie\_\-value}!cookies@{cookies}}\index{cookies@{cookies}!apreq_cookie_value@{apreq\_\-cookie\_\-value}}
\#define {\bf apreq\_\-cookie\_\-value}(c)\ ((c) $\rightarrow$ v.data)\label{group__cookies_a17}

\item 
\index{apreq_jar_items@{apreq\_\-jar\_\-items}!cookies@{cookies}}\index{cookies@{cookies}!apreq_jar_items@{apreq\_\-jar\_\-items}}
\#define {\bf apreq\_\-jar\_\-items}(j)\ {\bf apr\_\-table\_\-elts}(j $\rightarrow$ cookies) $\rightarrow$ nelts\label{group__cookies_a18}

\item 
\index{apreq_jar_nelts@{apreq\_\-jar\_\-nelts}!cookies@{cookies}}\index{cookies@{cookies}!apreq_jar_nelts@{apreq\_\-jar\_\-nelts}}
\#define {\bf apreq\_\-jar\_\-nelts}(j)\ {\bf apr\_\-table\_\-elts}(j $\rightarrow$ cookies) $\rightarrow$ nelts\label{group__cookies_a19}

\item 
\index{apreq_add_cookie@{apreq\_\-add\_\-cookie}!cookies@{cookies}}\index{cookies@{cookies}!apreq_add_cookie@{apreq\_\-add\_\-cookie}}
\#define {\bf apreq\_\-add\_\-cookie}(j, c)\ apreq\_\-jar\_\-add(j,c)\label{group__cookies_a20}

\item 
\index{apreq_make_cookie@{apreq\_\-make\_\-cookie}!cookies@{cookies}}\index{cookies@{cookies}!apreq_make_cookie@{apreq\_\-make\_\-cookie}}
\#define {\bf apreq\_\-make\_\-cookie}(p, n, nl, v, vl)\ apreq\_\-cookie\_\-make(p,n,nl,v,vl)\label{group__cookies_a21}

\item 
\index{apreq_serialize_cookie@{apreq\_\-serialize\_\-cookie}!cookies@{cookies}}\index{cookies@{cookies}!apreq_serialize_cookie@{apreq\_\-serialize\_\-cookie}}
\#define {\bf apreq\_\-serialize\_\-cookie}(buf, len, c)\ apreq\_\-cookie\_\-serialize(c,buf,len)\label{group__cookies_a22}

\end{CompactItemize}
\subsection*{Typedefs}
\begin{CompactItemize}
\item 
typedef apreq\_\-jar\_\-t {\bf apreq\_\-jar\_\-t}
\item 
typedef apreq\_\-cookie\_\-t {\bf apreq\_\-cookie\_\-t}
\end{CompactItemize}
\subsection*{Enumerations}
\begin{CompactItemize}
\item 
enum {\bf apreq\_\-cookie\_\-version\_\-t} \{ {\bf APREQ\_\-COOKIE\_\-VERSION\_\-NETSCAPE}, 
{\bf APREQ\_\-COOKIE\_\-VERSION\_\-RFC}
 \}
\end{CompactItemize}
\subsection*{Functions}
\begin{CompactItemize}
\item 
{\bf apreq\_\-cookie\_\-t} $\ast$ {\bf apreq\_\-cookie} (const {\bf apreq\_\-jar\_\-t} $\ast$jar, const char $\ast$name)
\item 
{\bf void} {\bf apreq\_\-jar\_\-add} ({\bf apreq\_\-jar\_\-t} $\ast$jar, const {\bf apreq\_\-cookie\_\-t} $\ast$c)
\item 
{\bf apreq\_\-jar\_\-t} $\ast$ {\bf apreq\_\-jar} ({\bf void} $\ast$env, const char $\ast$hdr)
\item 
{\bf apreq\_\-cookie\_\-t} $\ast$ {\bf apreq\_\-cookie\_\-make} ({\bf apr\_\-pool\_\-t} $\ast$p, const char $\ast$name, const {\bf apr\_\-size\_\-t} nlen, const char $\ast${\bf value}, const {\bf apr\_\-size\_\-t} vlen)
\item 
{\bf apr\_\-status\_\-t} {\bf apreq\_\-cookie\_\-attr} ({\bf apr\_\-pool\_\-t} $\ast$p, {\bf apreq\_\-cookie\_\-t} $\ast$c, const char $\ast$attr, {\bf apr\_\-size\_\-t} alen, const char $\ast$val, {\bf apr\_\-size\_\-t} vlen)
\item 
char $\ast$ {\bf apreq\_\-cookie\_\-as\_\-string} (const {\bf apreq\_\-cookie\_\-t} $\ast$c, {\bf apr\_\-pool\_\-t} $\ast$p)
\item 
{\bf int} {\bf apreq\_\-cookie\_\-serialize} (const {\bf apreq\_\-cookie\_\-t} $\ast$c, char $\ast$buf, {\bf apr\_\-size\_\-t} len)
\item 
{\bf void} {\bf apreq\_\-cookie\_\-expires} ({\bf apreq\_\-cookie\_\-t} $\ast$c, const char $\ast$time\_\-str)
\item 
{\bf apr\_\-status\_\-t} {\bf apreq\_\-cookie\_\-bake} (const {\bf apreq\_\-cookie\_\-t} $\ast$c, {\bf void} $\ast$env)
\item 
{\bf apr\_\-status\_\-t} {\bf apreq\_\-cookie\_\-bake2} (const {\bf apreq\_\-cookie\_\-t} $\ast$c, {\bf void} $\ast$env)
\item 
{\bf apreq\_\-cookie\_\-version\_\-t} {\bf apreq\_\-ua\_\-cookie\_\-version} ({\bf void} $\ast$env)
\end{CompactItemize}


\subsection{Define Documentation}
\index{cookies@{cookies}!APREQ_COOKIE_MAX_LENGTH@{APREQ\_\-COOKIE\_\-MAX\_\-LENGTH}}
\index{APREQ_COOKIE_MAX_LENGTH@{APREQ\_\-COOKIE\_\-MAX\_\-LENGTH}!cookies@{cookies}}
\subsubsection{\setlength{\rightskip}{0pt plus 5cm}\#define APREQ\_\-COOKIE\_\-MAX\_\-LENGTH\ 4096}\label{group__cookies_a14}


Maximum length of a single Set-Cookie(2) header \index{cookies@{cookies}!APREQ_COOKIE_VERSION_DEFAULT@{APREQ\_\-COOKIE\_\-VERSION\_\-DEFAULT}}
\index{APREQ_COOKIE_VERSION_DEFAULT@{APREQ\_\-COOKIE\_\-VERSION\_\-DEFAULT}!cookies@{cookies}}
\subsubsection{\setlength{\rightskip}{0pt plus 5cm}\#define APREQ\_\-COOKIE\_\-VERSION\_\-DEFAULT\ APREQ\_\-COOKIE\_\-VERSION\_\-NETSCAPE}\label{group__cookies_a13}


Default version, used when creating a new cookie. See {\bf apreq\_\-cookie\_\-make}() {\rm (p.\,\pageref{group__cookies_a5})}. \index{cookies@{cookies}!apreq_value_to_cookie@{apreq\_\-value\_\-to\_\-cookie}}
\index{apreq_value_to_cookie@{apreq\_\-value\_\-to\_\-cookie}!cookies@{cookies}}
\subsubsection{\setlength{\rightskip}{0pt plus 5cm}\#define apreq\_\-value\_\-to\_\-cookie(ptr)}\label{group__cookies_a15}


{\bf Value:}

\footnotesize\begin{verbatim}apreq_attr_to_type(apreq_cookie_t, \
                                                      v, ptr)\end{verbatim}\normalsize 


\subsection{Typedef Documentation}
\index{cookies@{cookies}!apreq_cookie_t@{apreq\_\-cookie\_\-t}}
\index{apreq_cookie_t@{apreq\_\-cookie\_\-t}!cookies@{cookies}}
\subsubsection{\setlength{\rightskip}{0pt plus 5cm}typedef struct apreq\_\-cookie\_\-t  apreq\_\-cookie\_\-t}\label{group__cookies_a1}


cookie XXX ... \index{cookies@{cookies}!apreq_jar_t@{apreq\_\-jar\_\-t}}
\index{apreq_jar_t@{apreq\_\-jar\_\-t}!cookies@{cookies}}
\subsubsection{\setlength{\rightskip}{0pt plus 5cm}typedef struct apreq\_\-jar\_\-t  apreq\_\-jar\_\-t}\label{group__cookies_a0}


Cookie Jar 

\subsection{Enumeration Type Documentation}
\index{cookies@{cookies}!apreq_cookie_version_t@{apreq\_\-cookie\_\-version\_\-t}}
\index{apreq_cookie_version_t@{apreq\_\-cookie\_\-version\_\-t}!cookies@{cookies}}
\subsubsection{\setlength{\rightskip}{0pt plus 5cm}enum apreq\_\-cookie\_\-version\_\-t}\label{group__cookies_a23}


Cookie Version. libapreq does not distinguish between rfc2109 and its successor rfc2965; both are referred to as APREQ\_\-COOKIE\_\-VERSION\_\-RFC. Users can distinguish between them in their outgoing cookies by using {\bf apreq\_\-cookie\_\-bake}() {\rm (p.\,\pageref{group__cookies_a10})} for sending rfc2109 cookies, or {\bf apreq\_\-cookie\_\-bake2}() {\rm (p.\,\pageref{group__cookies_a11})} for rfc2965. The original Netscape cookie spec is still preferred for its greater portability, it is named APREQ\_\-COOKIE\_\-VERSION\_\-NETSCAPE. 

\subsection{Function Documentation}
\index{cookies@{cookies}!apreq_cookie@{apreq\_\-cookie}}
\index{apreq_cookie@{apreq\_\-cookie}!cookies@{cookies}}
\subsubsection{\setlength{\rightskip}{0pt plus 5cm}{\bf apreq\_\-cookie\_\-t}$\ast$ apreq\_\-cookie (const {\bf apreq\_\-jar\_\-t} $\ast$ {\em jar}, const char $\ast$ {\em name})}\label{group__cookies_a2}


Fetches a cookie from the jar\begin{Desc}
\item[Parameters: ]\par
\begin{description}
\item[{\em 
jar}]The cookie jar. \item[{\em 
name}]The name of the desired cookie. \end{description}
\end{Desc}
\index{cookies@{cookies}!apreq_cookie_as_string@{apreq\_\-cookie\_\-as\_\-string}}
\index{apreq_cookie_as_string@{apreq\_\-cookie\_\-as\_\-string}!cookies@{cookies}}
\subsubsection{\setlength{\rightskip}{0pt plus 5cm}char$\ast$ apreq\_\-cookie\_\-as\_\-string (const {\bf apreq\_\-cookie\_\-t} $\ast$ {\em c}, {\bf apr\_\-pool\_\-t} $\ast$ {\em p})}\label{group__cookies_a7}


Returns a string that represents the cookie as it would appear  in a valid \char`\"{}Set-Cookie$\ast$\char`\"{} header.\begin{Desc}
\item[Parameters: ]\par
\begin{description}
\item[{\em 
c}]The cookie. \item[{\em 
p}]The pool. \end{description}
\end{Desc}
\index{cookies@{cookies}!apreq_cookie_attr@{apreq\_\-cookie\_\-attr}}
\index{apreq_cookie_attr@{apreq\_\-cookie\_\-attr}!cookies@{cookies}}
\subsubsection{\setlength{\rightskip}{0pt plus 5cm}{\bf apr\_\-status\_\-t} apreq\_\-cookie\_\-attr ({\bf apr\_\-pool\_\-t} $\ast$ {\em p}, {\bf apreq\_\-cookie\_\-t} $\ast$ {\em c}, const char $\ast$ {\em attr}, {\bf apr\_\-size\_\-t} {\em alen}, const char $\ast$ {\em val}, {\bf apr\_\-size\_\-t} {\em vlen})}\label{group__cookies_a6}


Sets the associated cookie attribute. \begin{Desc}
\item[Parameters: ]\par
\begin{description}
\item[{\em 
p}]Pool for allocating the new attribute. \item[{\em 
c}]Cookie. \item[{\em 
attr}]Name of attribute- leading '-' or '\$' characters are ignored. \item[{\em 
alen}]Length of attr. \item[{\em 
val}]Value of new attribute. \item[{\em 
vlen}]Length of new attribute. \end{description}
\end{Desc}
\begin{Desc}
\item[Remarks: ]\par
Ensures cookie version \& time are kept in sync. \end{Desc}
\index{cookies@{cookies}!apreq_cookie_bake@{apreq\_\-cookie\_\-bake}}
\index{apreq_cookie_bake@{apreq\_\-cookie\_\-bake}!cookies@{cookies}}
\subsubsection{\setlength{\rightskip}{0pt plus 5cm}{\bf apr\_\-status\_\-t} apreq\_\-cookie\_\-bake (const {\bf apreq\_\-cookie\_\-t} $\ast$ {\em c}, {\bf void} $\ast$ {\em env})}\label{group__cookies_a10}


Add the cookie to the outgoing \char`\"{}Set-Cookie\char`\"{} headers.\begin{Desc}
\item[Parameters: ]\par
\begin{description}
\item[{\em 
c}]The cookie. \end{description}
\end{Desc}
\index{cookies@{cookies}!apreq_cookie_bake2@{apreq\_\-cookie\_\-bake2}}
\index{apreq_cookie_bake2@{apreq\_\-cookie\_\-bake2}!cookies@{cookies}}
\subsubsection{\setlength{\rightskip}{0pt plus 5cm}{\bf apr\_\-status\_\-t} apreq\_\-cookie\_\-bake2 (const {\bf apreq\_\-cookie\_\-t} $\ast$ {\em c}, {\bf void} $\ast$ {\em env})}\label{group__cookies_a11}


Add the cookie to the outgoing \char`\"{}Set-Cookie2\char`\"{} headers.\begin{Desc}
\item[Parameters: ]\par
\begin{description}
\item[{\em 
c}]The cookie. \end{description}
\end{Desc}
\index{cookies@{cookies}!apreq_cookie_expires@{apreq\_\-cookie\_\-expires}}
\index{apreq_cookie_expires@{apreq\_\-cookie\_\-expires}!cookies@{cookies}}
\subsubsection{\setlength{\rightskip}{0pt plus 5cm}{\bf void} apreq\_\-cookie\_\-expires ({\bf apreq\_\-cookie\_\-t} $\ast$ {\em c}, const char $\ast$ {\em time\_\-str})}\label{group__cookies_a9}


Set the Cookie's expiration date.\begin{Desc}
\item[Parameters: ]\par
\begin{description}
\item[{\em 
c}]The cookie. \item[{\em 
time\_\-str}]If NULL, the Cookie's expiration date is unset, making it a session cookie. This means no \char`\"{}expires\char`\"{} or \char`\"{}max-age\char`\"{}  attribute will appear in the cookie's serialized form. If time\_\-str is not NULL, the expiration date will be reset to the offset (from now) represented by time\_\-str. The time\_\-str should be in a format that  {\bf apreq\_\-atoi64t}() {\rm (p.\,\pageref{group__Utils_a18})} can understand, namely /[+-]?$\backslash$d+$\backslash$s$\ast$[YMDhms]/. \end{description}
\end{Desc}
\index{cookies@{cookies}!apreq_cookie_make@{apreq\_\-cookie\_\-make}}
\index{apreq_cookie_make@{apreq\_\-cookie\_\-make}!cookies@{cookies}}
\subsubsection{\setlength{\rightskip}{0pt plus 5cm}{\bf apreq\_\-cookie\_\-t}$\ast$ apreq\_\-cookie\_\-make ({\bf apr\_\-pool\_\-t} $\ast$ {\em pool}, const char $\ast$ {\em name}, const {\bf apr\_\-size\_\-t} {\em nlen}, const char $\ast$ {\em value}, const {\bf apr\_\-size\_\-t} {\em vlen})}\label{group__cookies_a5}


Returns a new cookie, made from the argument list.\begin{Desc}
\item[Parameters: ]\par
\begin{description}
\item[{\em 
pool}]Pool which allocates the cookie. \item[{\em 
name}]The cookie's name. \item[{\em 
nlen}]Length of name. \item[{\em 
value}]The cookie's value. \item[{\em 
vlen}]Length of value. \end{description}
\end{Desc}
\index{cookies@{cookies}!apreq_cookie_serialize@{apreq\_\-cookie\_\-serialize}}
\index{apreq_cookie_serialize@{apreq\_\-cookie\_\-serialize}!cookies@{cookies}}
\subsubsection{\setlength{\rightskip}{0pt plus 5cm}{\bf int} apreq\_\-cookie\_\-serialize (const {\bf apreq\_\-cookie\_\-t} $\ast$ {\em c}, char $\ast$ {\em buf}, {\bf apr\_\-size\_\-t} {\em len})}\label{group__cookies_a8}


Same functionality as apreq\_\-cookie\_\-as\_\-string. Stores the string representation in buf, using up to len bytes in buf as storage. The return value has the same semantics as that of apr\_\-snprintf, including the special behavior for a \char`\"{}len = 0\char`\"{} argument.\begin{Desc}
\item[Parameters: ]\par
\begin{description}
\item[{\em 
c}]The cookie. \item[{\em 
buf}]Storage location for the result. \item[{\em 
len}]Size of buf's storage area. \end{description}
\end{Desc}
\index{cookies@{cookies}!apreq_jar@{apreq\_\-jar}}
\index{apreq_jar@{apreq\_\-jar}!cookies@{cookies}}
\subsubsection{\setlength{\rightskip}{0pt plus 5cm}{\bf apreq\_\-jar\_\-t}$\ast$ apreq\_\-jar ({\bf void} $\ast$ {\em env}, const char $\ast$ {\em hdr})}\label{group__cookies_a4}


Parse the incoming \char`\"{}Cookie:\char`\"{} headers into a cookie jar.\begin{Desc}
\item[Parameters: ]\par
\begin{description}
\item[{\em 
env}]The current environment. \item[{\em 
hdr}]String to parse as a HTTP-merged \char`\"{}Cookie\char`\"{} header. \end{description}
\end{Desc}
\begin{Desc}
\item[Remarks: ]\par
\char`\"{}data = NULL\char`\"{} has special behavior. In this case, apreq\_\-jar(env,NULL) will attempt to fetch a cached object from the environment via apreq\_\-env\_\-jar. Failing that, it will replace \char`\"{}hdr\char`\"{} with the result of {\bf apreq\_\-env\_\-cookie}(env) {\rm (p.\,\pageref{group__ENV_a14})}, parse that, and store the resulting object back within the environment. This maneuver is designed to mimimize parsing work, since generating the cookie jar is relatively expensive. \end{Desc}
\index{cookies@{cookies}!apreq_jar_add@{apreq\_\-jar\_\-add}}
\index{apreq_jar_add@{apreq\_\-jar\_\-add}!cookies@{cookies}}
\subsubsection{\setlength{\rightskip}{0pt plus 5cm}{\bf void} apreq\_\-jar\_\-add ({\bf apreq\_\-jar\_\-t} $\ast$ {\em jar}, const {\bf apreq\_\-cookie\_\-t} $\ast$ {\em c})}\label{group__cookies_a3}


Adds a cookie by pushing it to the bottom of the jar.\begin{Desc}
\item[Parameters: ]\par
\begin{description}
\item[{\em 
jar}]The cookie jar. \item[{\em 
c}]The cookie to add. \end{description}
\end{Desc}
\index{cookies@{cookies}!apreq_ua_cookie_version@{apreq\_\-ua\_\-cookie\_\-version}}
\index{apreq_ua_cookie_version@{apreq\_\-ua\_\-cookie\_\-version}!cookies@{cookies}}
\subsubsection{\setlength{\rightskip}{0pt plus 5cm}{\bf apreq\_\-cookie\_\-version\_\-t} apreq\_\-ua\_\-cookie\_\-version ({\bf void} $\ast$ {\em env})}\label{group__cookies_a12}


Looks for the presence of a \char`\"{}Cookie2\char`\"{} header to determine whether or not the current User-Agent supports rfc2965. \begin{Desc}
\item[Parameters: ]\par
\begin{description}
\item[{\em 
env}]The current environment. \end{description}
\end{Desc}
\begin{Desc}
\item[Returns: ]\par
APREQ\_\-COOKIE\_\-VERSION\_\-RFC if rfc2965 is supported,  APREQ\_\-COOKIE\_\-VERSION\_\-NETSCAPE otherwise. \end{Desc}
